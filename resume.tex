% resume.tex
% vim:set ft=tex spell:
\documentclass[12pt,a4paper]{article}
\usepackage[a4paper,margin=0.8in]{geometry}
\usepackage{mdwlist}
\usepackage{roboto}  %% Option 'sfdefault' only if the base font of the document is to be sans serif
\usepackage[T1]{fontenc}
\usepackage{textcomp}
\usepackage[dvipsnames]{xcolor}
\usepackage{hyperref}
\usepackage{multirow}
\usepackage{tabu}

\pagestyle{empty}
\setlength{\tabcolsep}{0em}

% make "C++" look pretty when used in text by touching up the plus signs
\newcommand{\CPP}
{C\nolinebreak[4]\hspace{-.05em}\raisebox{.22ex}{\footnotesize\bf ++}}

% format two pieces of text, one left aligned and one right aligned
\newcommand{\headerrow}[2]
{\begin{tabular*}{\linewidth}{l@{\extracolsep{\fill}}r}
	#1 &
	#2 \\
\end{tabular*}}

% put some color into your dividers and make them a bit thicker than default
\newcommand{\colorrule}[1]
{
  {\color{#1}\hrule height 2pt}
  \vspace{1.0em}
}


\begin{document}
\begin{center}
	\begin{tabu} to 0.9\textwidth {X[l] X[r]}
	  \multirow{2}{*}{\LARGE \textbf{Frank Olaf Berghaus}} & {\small Rue de la Servette 78} \\
																				             	  & {\small CH-1202 Genève} \\
																					              & {\small +41 (0)77 995 0502} \\
	  {\small Nationanlity: {\sc German \& Canadian}}		  & {\small frank.berghaus@gmail.com} \\
	\end{tabu}
\end{center}

\colorrule{NavyBlue}
\subsection*{Experience}
\begin{itemize}
	\parskip=0.1em

	\item
	\headerrow
		{\textbf{CERN}}
		{\textbf{Meyrin, Genève}}
	\\
	\headerrow
		{\emph{Fellow on Data Preservation in HEP}}
		{\emph{2015 -- presnt}}
	\begin{itemize*}
		\item Preserving CERN Program Library documentation
		\item Implementing an environment for LEP software preservation
    \item Built the collaboration \href{http://dphep.web.cern.ch}{website}
		\item Learned the requirements for digital archives by RDA and ISO standards
		\item Evaluated the preservation technologies in place at CERN
	\end{itemize*}

	\item
	\headerrow
		{\textbf{University of Victoria}}
		{\textbf{Victoria, BC, Canada}}
	\\
	\headerrow
		{\emph{Research Associate}}
		{\emph{2013 -- 2014}}
	\begin{itemize*}
		\item Developed of cloud and data federation software
    \item Deployed of elastic batch computing resources for High Energy
      Physics experiments
    \item Included of Nimbus, OpenStack, Amazon EC2, and Google Compute Engine
     in ATLAS and Belle II computing
    \item Integrated cloud resources into the PanDA and DIRAC workload
      management systems
	\end{itemize*}
\end{itemize}


\colorrule{NavyBlue}
\subsection*{Education}
\begin{itemize}
  \parskip=0.1em

	\item
	\headerrow
		{\textbf{University of Victoria}}
		{\textbf{Victoria, British Columbia}}
	\\
	\headerrow
		{\emph{Faculty of Graduate Studies, Doctor of Philosophy}}
		{\emph{2006 -- 2013}}

  \item
	\headerrow
		{\textbf{University of British Columbia}}
		{\textbf{Vancouver, British Columbia}}
	\\
	\headerrow
		{\emph{Faculty of Graduate Studies, Master of Science}}
		{\emph{2003 -- 2006}}

  \item
	\headerrow
		{\textbf{Saint Mary's University}}
		{\textbf{Halifax, Nova Scotia}}
	\\
	\headerrow
		{\emph{Department of Physics and Astronomy, Bachelor of Science}}
		{\emph{1999 -- 2003}}

\end{itemize}

\subsubsection*{PhD Dissertation}
\begin{itemize}
	\item[Title]
	{\it Search for Quark Compositeness in 7 TeV Proton-Proton Collisions with the
	 ATLAS Detector at the Large Hadron Collider}
	\item[Committee]
	{\bf Dr.\ Michel Lefebvre}, Dr.\ Rob McPherson, Dr.\ Randall Sobie,
	Dr.\ Stan Dosso
	\item[Abstract]
	{\small The ATLAS detector at CERN’s Large Hadron Collider is collecting
	 data to investigate proton collisions at the highest energy ever produced. My
	 thesis searches for structure in quarks by measuring the inclusive dijet
	 spectrum with the ATLAS detector.}
\end{itemize}

\subsubsection*{Masters Thesis}
\begin{itemize}
	\item[Title]
	{\it K2K Near Detector Laserball Calibration: Manipulator Motivation, Design
	 and Results}
	\item[Supervisor]
	{\bf Dr.\ Scott Oser}
	\item[Abstract]
	{\small The KEK to Kamiokande (K2K) experiment uses a muon anti-neutrino
	($\bar{\nu}_\mu$) beam generated at the KEK accelerator facility aimed at the
	Super-Kamiokande detector to measure ‹μ oscillations. The beam first traverses
	a Near Detector at KEK and later the Super-Kamiokande detector 250km away. The
	$\bar{\nu}_\mu$ oscillation is inferred by the disappearance of muon
	anti-neutrinos. My thesis exploits the entire volume of the Near Detector to
	calibrate for neutrino detection.}
\end{itemize}

\subsubsection*{Awards}



\colorrule{NavyBlue}
\subsection*{Publications}

\colorrule{NavyBlue}
\subsection*{Presentations}

\end{document}
