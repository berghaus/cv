% resume.tex
% vim:set ft=tex spell:
\documentclass[12pt,a4paper]{article}
\usepackage[a4paper,margin=0.8in]{geometry}
\usepackage{mdwlist}
\usepackage{roboto}  %% Option 'sfdefault' only if the base font of the document is to be sans serif
\usepackage[T1]{fontenc}
\usepackage{textcomp}
\usepackage[dvipsnames]{xcolor}
\usepackage{hyperref}
\usepackage{multirow}
\usepackage{tabu}
\usepackage{natbib}
\usepackage{bibentry}
\nobibliography*

\pagestyle{empty}
\setlength{\tabcolsep}{0em}

% make "C++" look pretty when used in text by touching up the plus signs
\newcommand{\CPP}
{C\nolinebreak[4]\hspace{-.05em}\raisebox{.22ex}{\footnotesize\bf ++}}

% format two pieces of text, one left aligned and one right aligned
\newcommand{\headerrow}[2]
{\begin{tabular*}{\linewidth}{l@{\extracolsep{\fill}}r}
	#1 &
	#2 \\
\end{tabular*}}

% put some color into your dividers and make them a bit thicker than default
\newcommand{\colorrule}[1]
{
  {\color{#1}\hrule height 2pt}
  \vspace{1.0em}
}

% create some space between cells in a row
\setlength{\tabcolsep}{5pt}

\begin{document}
\begin{center}
	\begin{tabu} to 0.9\textwidth {X[l] X[r]}
	  \multirow{2}{*}{\LARGE \textbf{Frank Olaf Berghaus}}
		                                      & {\small Rue de la Servette 78} \\
																          & {\small CH-1202 Geneva} \\
																          & \href{http://www.frankberghaus.com}{\small frankberghaus.com} \\
	                                    		& {\small frank.berghaus@cern.ch} \\
	\end{tabu}
\end{center}
\colorrule{NavyBlue}
Frank is a physicist and computing expert with experience in the long-term
preservation of scientific data and usage of cloud computing technology for
high throughput scientific computing.

\subsection*{Experience}
\begin{itemize}
	\parskip=0.1em

	\item
	\headerrow
		{\textbf{CERN}}
		{\textbf{Geneva, Switzerland}}
	\\
	\headerrow
		{\emph{Fellow on Data Preservation in HEP}}
		{\emph{2015 -- present}}
	\begin{itemize*}
		\item Preserving CERNLIB and LEP documentation, computing, and data
    \item Built the DPHEP \href{http://cern.ch/dphep}{portal}
		\item Evaluated the preservation technologies in place at CERN under ISO
			standards
		\item Developed data management and preservation plans and strategies
		\item Established a workflow for users to retrieve LEP and OPERA data
			archived on CASTOR for analysis
	\end{itemize*}

	\item
	\headerrow
		{\textbf{University of Victoria}}
		{\textbf{Victoria, BC, Canada}}
	\\
	\headerrow
		{\emph{Research Associate}}
		{\emph{2013 -- 2014}}
	\begin{itemize*}
		\item Developed cloud and data federation software
    \item Deployed elastic batch computing resources for High Energy
      Physics experiments
    \item Included OpenStack, Amazon, and Google Compute Engine
     in ATLAS and Belle II computing systems
    \item Integrated cloud resources into the PanDA and DIRAC workload
      management systems
	\end{itemize*}

	\item
	\headerrow
		{\textbf{University of Victoria}}
		{\textbf{Victoria, BC, Canada}}
	\\
	\headerrow
		{\emph{Research Associate}}
		{\emph{2006 -- 2013}}
	\begin{itemize*}
		\item Analysed ATLAS data employing a profile likelihood approach
		\item Developed jet calibration procedures for the ATLAS experiment
		\item Developed monitoring software for the liquid argon calorimeter
	\end{itemize*}
\end{itemize}

\subsubsection*{Skills}
\begin{tabu} to \textwidth{r p{6.5cm} r p{6.5cm}}
	Design & Programming patterns and practices & \textsc{Linux}  & 17 years user
		and four years administrative experience \\
  \rule{0pt}{4ex}
	\CPP   & Nine years programming experience     & \textsc{Python}   & Three
		years of programming experience \\
	\rule{0pt}{4ex}
	git    & Collaborative code development     & \LaTeX & Writing scientific
	 documents and presentations \\
	\rule{0pt}{4ex}
	Clouds & Usage of OpenStack, Nimbus, Amazon EC2, and GCE & Archival & Design
	and assessment of in Open Archival Information Systems\\
	\rule{0pt}{4ex}
\end{tabu}

\colorrule{NavyBlue}
\subsection*{Education}
\begin{itemize}
  \parskip=0.1em

	\item
	\headerrow
		{\textbf{University of Victoria}}
		{\textbf{Victoria, BC, Canada}}
	\\
	\headerrow
		{\emph{Faculty of Graduate Studies, Doctor of Philosophy}}
		{\emph{2006 -- 2013}}

  \item
	\headerrow
		{\textbf{University of British Columbia}}
		{\textbf{Vancouver, BC, Canada}}
	\\
	\headerrow
		{\emph{Faculty of Graduate Studies, Master of Science}}
		{\emph{2003 -- 2006}}

  \item
	\headerrow
		{\textbf{Saint Mary's University}}
		{\textbf{Halifax, NS, Canada}}
	\\
	\headerrow
		{\emph{Department of Physics and Astronomy, Bachelor of Science}}
		{\emph{1999 -- 2003}}

\end{itemize}

\subsubsection*{PhD Dissertation}
\begin{tabular}{r p{14.3cm}}
	Title     & {\it Search for Quark Compositeness in 7 TeV Proton-Proton
							 Collisions with the ATLAS Detector at the Large Hadron
							 Collider} \\
	Committee & {\bf Dr.\ Michel Lefebvre}, Dr.\ Rob McPherson,
							 Dr.\ Randall Sobie, Dr.\ Stan Dosso \\
	% Abstract  & {\small The ATLAS detector at CERN's Large Hadron Collider is
	%              collecting data to investigate proton collisions at the highest
	% 						 energy ever produced. My thesis searches for structure in quarks
	% 						 by measuring the inclusive dijet spectrum with the ATLAS
	% 						 detector.} \\
\end{tabular}

\subsubsection*{Masters Thesis}
\begin{tabular}{r p{14.3cm}}
	Title & {\it K2K Near Detector Laserball Calibration: Manipulator Motivation, Design
	 				 and Results} \\
	Supervisor & {\bf Dr.\ Scott Oser} \\
% 	\item[Abstract]
% 	{\small The KEK to Kamiokande (K2K) experiment uses a muon anti-neutrino
% 	($\bar{\nu}_\mu$) beam generated at the KEK accelerator facility aimed at the
% 	Super-Kamiokande detector to measure ‹μ oscillations. The beam first traverses
% 	a Near Detector at KEK and later the Super-Kamiokande detector 250km away. The
% 	$\bar{\nu}_\mu$ oscillation is inferred by the disappearance of muon
% 	anti-neutrinos. My thesis exploits the entire volume of the Near Detector to
% 	calibrate for neutrino detection.}
\end{tabular}


\subsubsection*{Training}
\begin{itemize}
	\item[2016] \emph{CERN School of Computing}, {CEN-SCK}, Mol,
		\textsc{Belgium}
	\item[2016] \emph{CERN Inverted School of Computing},  {CERN}, Geneva,
		\textsc{Switzerland}\\
	{\small Scientific Computing and Programming}
	\item[2015] \emph{Training on ISO 16363}, {CERN}, Geneva, \textsc{Switzerland}\\
	{\small Audit and certification of trustworthy digital repositories}
	% \item[2015] \emph{Language Training}, {CERN}, Geneva, \textsc{Switzerland}\\
	% {\small French A1 \& A2}
	% \item[2012] \emph{Pathways to Success}, {University of Victoria},
	% Victoria, \textsc{Canada}\\
	% {\small Professional Development}
	% \item[2009] \emph{CTEQ Summer School}, {University of
	% Wisconsin-Madison}, Madison, \textsc{USA}\\
	% {\small Perturbative QCD}
	% \item[2006] \emph{TRIUMF Summer Insitute}, {TRIUMF}, Vancouver,
	% \textsc{Canada}\\
	% {\small Particle Physics}
\end{itemize}

\subsubsection*{Awards}
\begin{itemize}
	\small
	\setlength\itemsep{-0.5em}
	\item[2016]
	Graduate with ``special distinction'', CERN School of Computing.
	\item[2010]
	Eric Foster Graduate Scholarship in Physics, University of Victoria.
	\item[2005]
	UBC Award for Teaching Excellence as Teaching Assistant, University of British
	Columbia.
	\item[2003]
	Graduate Entrance Scholarship, University of British Columbia.
	\item[2002]
	TRIUMF Research Fellowship, University of British Columbia.
	\item[2002]
	Dr. C. Henry Reardon Scholarship, Saint Mary's University.
	\item[2002]
	The Monsignor Richard J. Murphy Scholarship, Saint Mary's University.
	\item[2001]
	NSERC Undergraduate Student Research Assistant Fellowship, Saint Mary's
	University.
	\item[2001]
	Shatford Trust, Saint Mary's University.
	\item[2000]
	First Place APICS Mathematics Competition, Atlantic Provinces Council on the
	Sciences.
	\item[2000]
	Achievement Scholarship until 2003, Saint Mary's University.
\end{itemize}


\colorrule{NavyBlue}
\subsection*{Selected Publications}
%\bibliographystyle{acm}
%remove "References" from bibliography header
%{\def\section*#1{}\bibliography{publications}}
%\nocite{*}
\bibliographystyle{plain}
\nobibliography{publications}
\begin{itemize}
	\item[2015] \bibentry{DPHEPStatusReport}
	\item[2014] \bibentry{1742-6596-513-3-032035}
	\item[2010] \bibentry{Aad:2010ai}
	%\item[2006] \bibentry{Morris:2006df}
\end{itemize}


\colorrule{NavyBlue}
\subsection*{Selected Presentations}
\begin{itemize}
	\item[2016] \emph{DPHEP Portal \& LEP Progress}. WLCG/DPHEP Workshop.  Feb
		2014. Lisbon, Portugal.
	\item[2014] \emph{Cloud Operations and Integration}. ATLAS Software and
		Computing Workshop. Feb 2014. CERN, Geneva.
	\item[2012] \emph{Search for Quark Substructure in 7 TeV Centre-of-Mass
		Proton-Proton Collisions with the ATLAS Detector at the LHC}. CAP Congress.
		Jun 2012. University of Calgary, Calgary.
\end{itemize}
\end{document}
